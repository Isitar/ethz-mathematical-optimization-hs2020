\documentclass[12pt, a4]{article}
\usepackage[margin=2cm]{geometry}
\usepackage{parskip}
\usepackage{nameref}
\usepackage{enumitem}
\usepackage{tabularx}
\usepackage[tiny]{titlesec}
\usepackage{amsmath}
\usepackage{amssymb}
\usepackage{mathtools}
\usepackage{bm}
\usepackage{fancyhdr}
\usepackage{titling}
\usepackage[]{hyperref}
\usepackage[nameinlink]{cleveref}

\author{Pascal Lüscher}
\title{Simplex tableau exchange steps}

\makeatletter
\let\mytitle\@title
\makeatother

\pagestyle{fancy}
\fancyhf{}
\rhead{
	\mytitle\\
	\theauthor
}

\rfoot{
	Page: \thepage
}

\renewcommand{\arraystretch}{1.2} % more space in tables
\renewcommand\thesubsection{\thesection.\alph{subsection}}

\newcolumntype{L}{>{$}l<{$}} % math-mode version of "l" column type
\newcolumntype{R}{>{$}r<{$}} % math-mode version of "r" column type
\newcolumntype{C}{>{$}c<{$}} % math-mode version of "c" column type

\begin{document}
	\section{Short tableau from canonical form}
	Given the an lp in canonical form like \cref{eq:lp} the short tableau is created simply reading the values. The Equation \cref{eq:example_lp} is an example which is transformed into the short tableau in \cref{tab:short_tableau_example_lp}.
	
\begin{minipage}[t]{.5\textwidth}
	\begin{equation}{\label{eq:lp}}
		\begin{matrix*}[r]
			\text{max} & c^\top x \\
			\text{s.t.} & Ax & \leq & b \\
			& x & \in& \mathbb{R}_{\geq 0}
		\end{matrix*}
	\end{equation}
\end{minipage}
%
\begin{minipage}[t]{.5\textwidth}
	\begin{equation}{\label{eq:example_lp}}
		\begin{matrix*}[r]
			\text{max} & & x_1 &+& x_2 \\
			\text{s.t.}& & x_1 & &     & \leq & 4 \\
			           & &     & & x_2 & \leq & 4 \\
			           & & x_1 &+& x_2 & \leq & 7 \\
			           &-& x_1 &-& x_2 & \leq &-3 \\		           
	          		   & & x_1 &,& x_2     &  \in & \mathbb{R}_{\geq 0} \\
		\end{matrix*}
	\end{equation}
\end{minipage}

\begin{table}[h]
	\centering
	\begin{tabular}{L|RR||R}
		& x_1 & x_2 & 1 \\
		\hline
		z & -1 & -1 & 0 \\
		\hline
		y_1 & 1 & 0 & 4 \\
		y_2 & 0 & 1 & 4 \\
		y_3 & 1 & 0 & 4 \\
		y_4 & 1 & 1 & 7 \\
		y_5 &-1 &-1 &-3 \\
	\end{tabular}
	\caption{Short tableau for \cref{eq:example_lp}}
	\label{tab:short_tableau_example_lp}
\end{table}

\section{Exchange step rules for short tableau}
We pivot on the element $A_{p_rp_c}$.

Pivot element
\begin{equation}{\label{eq:pivot_element}}
	A'_{p_rp_c} = \frac{1}{A_{p_rp_c}}
\end{equation}

Pivot row
\begin{equation}{\label{eq:pivot_row}}
	A'_{p_rc} = \frac{A_{p_rc}}{A_{p_rp_c}}
\end{equation}

Pivot column
\begin{equation}{\label{eq:pivot_column}}
	A'_{rp_c} = -\frac{A_{rp_c}}{A_{p_rp_c}}
\end{equation}

other A
\begin{equation}{\label{eq:other_a}}
	A'_{rc} = A_{rc} - \frac{A_{rp_c}A_{p_rc}}{A_{p_rp_c}}
\end{equation}

pivot row b
\begin{equation}{\label{eq:pivot_b}}
	b'_{p_r} = \frac{b_{p_r}}{A_{p_rp_c}}
\end{equation}
other  b
\begin{equation}{\label{eq:other_b}}
	b'_{r} = b_r - \frac{A_{rp_c} b_{p_r}}{A_{p_rp_c}}
\end{equation}

\end{document}