\documentclass[12pt, a4]{article}
\usepackage[margin=2cm]{geometry}
\usepackage{parskip}
\usepackage{nameref}
\usepackage{enumitem}
\usepackage{tabularx}
\usepackage{hyperref}
\usepackage[tiny]{titlesec}

\usepackage{amsmath}
\usepackage{amssymb}

\usepackage{fancyhdr}
\usepackage{titling}

\usepackage{pgfplots}
\pgfplotsset{compat=1.16}
\usetikzlibrary{decorations.pathreplacing,positioning}


\usepackage{xcolor}
\usepackage{graphicx}
\usepackage{fancyvrb}
\usepackage{listings}
\usepackage{bm}
\usepackage{xcolor}
\usepackage{optidef}


\xdefinecolor{gray}{rgb}{0.4,0.4,0.4}
\xdefinecolor{blue}{RGB}{58,95,205}
\xdefinecolor{darkgreen}{RGB}{0,100,0}

\newcommand{\lightgray}{black!30}

\newcommand{\plotDomain}{-1:10}

\newcommand{\addPlotLDownCoords}[1]{
	\addplot[mark=none, domain=\plotDomain, color=\lightgray,
	decoration={border,segment length=1mm,amplitude=1.5mm,angle=-135},
	postaction={decorate}
	] coordinates {#1};
	\addplot[mark=none, domain=\plotDomain] coordinates {#1};
}

\newcommand{\addPlotLDown}[1]{
	\addplot[mark=none, domain=\plotDomain, color=\lightgray,
	decoration={border,segment length=1mm,amplitude=1.5mm,angle=-135},
	postaction={decorate}
	] {#1};
	\addplot[mark=none, domain=\plotDomain] {#1};
}

\newcommand{\addPlotRUpCoords}[1]{
	\addplot[mark=none, domain=\plotDomain, color=\lightgray,
	decoration={border,segment length=1mm,amplitude=1.5mm,angle=135},
	postaction={decorate}
	] coordinates {#1};
	\addplot[mark=none, domain=\plotDomain] coordinates {#1};
}

\newcommand{\addPlotRUp}[1]{
	\addplot[mark=none, domain=\plotDomain, color=\lightgray,
	decoration={border,segment length=1mm,amplitude=1.5mm,angle=135},
	postaction={decorate}
	] {#1};
	\addplot[mark=none, domain=\plotDomain] {#1};
}

\lstset{% setup listings
	language=R,% set programming language
	basicstyle=\ttfamily\small,% basic font style
	keywordstyle=\color{blue},% keyword style
	commentstyle=\color{gray},% comment style
	numbers=left,% display line numbers on the left side
	numberstyle=\scriptsize,% use small line numbers
	numbersep=10pt,% space between line numbers and code
	tabsize=3,% sizes of tabs
	showstringspaces=false,% do not replace spaces in strings by a certain character
	captionpos=b,% positioning of the caption below
	breaklines=true,% automatic line breaking
	escapeinside={(*}{*)},% escaping to LaTeX
	fancyvrb=true,% verbatim code is typset by listings
	extendedchars=false,% prohibit extended chars (chars of codes 128--255)
	literate={"}{{\texttt{"}}}1{<-}{{$\bm\leftarrow$}}1{<<-}{{$\bm\twoheadleftarrow$}}1
	{~}{{$\bm\sim$}}1{<=}{{$\bm\le$}}1{>=}{{$\bm\ge$}}1{!=}{{$\bm\neq$}}1{^}{{$^{\bm\wedge}$}}1,% item to replace, text, length of chars
	alsoletter={.<-},% becomes a letter
	alsoother={$},% becomes other
	otherkeywords={!=, ~, $, \&, \%/\%, \%*\%, \%\%, <-, <<-, /},% other keywords
	deletekeywords={c}% remove keywords
}



\author{Pascal Lüscher}
\title{Mathematical Optimization – Problem set 4}

\makeatletter
\let\mytitle\@title
\makeatother

\pagestyle{fancy}
\fancyhf{}
\rhead{
	\mytitle\\
	\theauthor
}

\rfoot{
	Page: \thepage
}

\renewcommand{\arraystretch}{1.2} % more space in tables
\renewcommand\thesubsection{\thesection.\alph{subsection}}
\titleformat{\section}[hang]{\normalfont\bfseries\itshape}{\textup{\thesubsection}}{1em}{}

\titleformat{\subsection}[hang]{\normalsize\itshape}{\textup{\thesubsection}}{1em}{}[]

\newcommand{\norm}[1]{\lvert #1 \rvert}

\newcolumntype{L}{>{$}l<{$}} % math-mode version of "l" column type
\newcolumntype{R}{>{$}r<{$}} % math-mode version of "l" column type
\newcolumntype{C}{>{$}c<{$}} % math-mode version of "l" column type


\begin{document}
	\section{Problem 1: Linear program with unique optimal solution}
For a given LP, suppose that there exists a feasible short tableau of the form
\begin{center}
	\begin{tabular}{c|rcr|r}
		&$x_1 $&$ \ldots$&$x_n$ & $1$\\
		\hline
		$z$ & $c_1$ & $\ldots$ & $c_n$ & $q$ \\
		\hline
		$x_{n+1}$ & $A_{11}$ & $\ldots$ & $A_{1n}$ & $b$ \\
		$\vdots$ & $\vdots$ && $\vdots$ & $\vdots$ \\
		$x_{n+m} $ & $A_{m1}$ & $\ldots$ & $A_{mn}$ & $b_m$	
	\end{tabular}
\end{center}
	such that $c_i>0$ for every $i\in \{1, \ldots , n\}$, i.e., all tableau entries that are objective function coefficients (excluding the current basic solution value $q$) are strictly positive. Prove that in this case, the linear program has a unique optimal solution.

\subsection{Proof}
We already know by Theorem 1.68 that a feasible tableau with $c_k \geq 0 \quad\forall k \in [n]$ is an optimal solution to the corresponding lp. So we only need to proof it is the unique solution and there exists no other solution with the same objective value.
If there were another solution with the same objective value, there would be a legal pivot from one solution to the other. But by the definition given, there is no $c_k <0$, so there is no legal pivot.

\section{Problem 2 Reading simplex tableaus}

\xdefinecolor{legalPivotColor}{RGB}{200,100,0}
\newcommand{\legalPivot}[1]{{\color{legalPivotColor}\fbox{#1}}}
	
Consider the short tableaus (a)–(d) given below. For each of them, answer the following questions and argue why your answer is correct.
\begin{itemize}
	\item {\color{legalPivotColor} Mark all entries that are legal pivots according to the rules of phase II of the Simplex Method.}
	\item What is the basic solution corresponding to the tableau? Is it feasible? If yes, can you determine if it is optimal by looking at the tableau only?
	\item Can you decide if the objective function of the linear program corresponding to the tableau is unbounded by looking at the tableau only? If it is unbounded, provide a feasible ray along which the objective function is strictly increasing.
\end{itemize}


\begin{minipage}[h]{.5\textwidth}
	\begin{enumerate}
		\item[(a)] {\begin{tabular}{C|RRR|R}
				&x_1&x_0&x_3 & 1\\
				\hline
				z&\frac{3}{8}&0&1&-12 \\
				\hline
				x_4 & 1 & 0 & -2 & 6 \\
				x_2 & -\frac{3}{4} & 7 & -3 & 1 \\
				x_5 & \frac{125}{8} & 10 & -3 & 0
		\end{tabular}}
	\end{enumerate}
\end{minipage}
\begin{minipage}[h]{.5\textwidth}
	\begin{enumerate}
		\item[(b)] {\begin{tabular}{C|RRRRRR|R}
				& y_1 & y_2 & x_s^+ & x_s^{-} & x_b & x_t & 1 \\
				\hline
				z& 0 & -\frac{4}{19} & 0 & 5 & -3 & -1 & 12 \\
				\hline
				y_3 & 1 & \legalPivot{2} & -\frac{5}{2} & 7 & 0 & -2 & 2 \\
				x_a & -3 & -1 & \frac{3}{2} & 0 & -3 & 0 & 0 \\
				y_4 & 0 & 0 & \frac{1}{2} & 0 & -\frac{1}{10} & -1 & 5
		\end{tabular}}
	\end{enumerate}
\end{minipage}
\begin{minipage}[h]{.5\textwidth}
	\begin{enumerate}
		\item[(c)] {\begin{tabular}{C|RRRRR|R}
				& J & A & B & G & D & 1 \\
				\hline
				z& -4 & 6 & 0 & -1 & -\frac{1}{2} & 0\\
				\hline
				C & \legalPivot{6} & -2 & -5 & 0 & -7 & 0 \\
				F & -3 & -\frac{17}{2} & 3 & \legalPivot{2} & 0 & 4 \\
				H & 3 & 91 & -3 & 4 & -20 & 12 \\
				E & 0 & 14 & \frac{7}{2} & \legalPivot{$\frac{1}{2}$} & \legalPivot{2} & 1	
		\end{tabular}}
	\end{enumerate}
\end{minipage}
\begin{minipage}[h]{.5\textwidth}
	\begin{enumerate}
		\item[(d)] {\begin{tabular}{C|RRR|R}
				& y_1 & y_2 & x_3 & 1 \\
				\hline
				z& 3 & -5 & 0 & 4 \\
				\hline
				y_3 & 2 & 0 & -2 & 2 \\
				x_1 & 3 & \legalPivot{7} & 3 & 1 \\
				x_2 & 1 & -1 & -3 & \frac{9}{2} \\
				y_4 & 0 & 10 & 3 & 5
		\end{tabular}}
	\end{enumerate}
\end{minipage}

\subsection{What is the basic solution corresponding to the tableau?}
\begin{enumerate}[label=(\alph*)]
	\item $x^T = (0,0,1,0,6,0)$
	\item $y_3 = 2, y_4 = 5$ rest is $0$
	\item $F = 4, H = 12, E = 1$ rest is $0$
	\item $y_3 =2, x_1 = 1, x_2 = \frac{9}{2}, y_4 = 5$ rest is $0$
\end{enumerate}
\subsection{Is it feasible? If yes, can you determine if it is optimal by looking at the tableau only?}
\begin{enumerate}[label=(\alph*)]
	\item it is feasible ($\text{rhs} \geq 0$) and optimal ($c_k \geq 0 \quad\forall k \ in [n]$)
	\item it is feasible. it is not optimal. 
	\item it is feasible, Unclear if optimal, there are negative coefficients. May be an optimal solution but certainly not the unique optimal solution.
	\item Same as in (c)
\end{enumerate}
	
\subsection{Can you decide if the objective function of the linear program corresponding to the tableau is unbounded by looking at the tableau only? If it is unbounded, provide a feasible ray along which the objective function is strictly increasing.}
Yes, if there is a column $k$ where $c_k < 0$ and $A_{jk} \leq 0 \forall j \in [m]$ like the column $x_b$ in tableau (b).
The feasible ray with variables $\begin{pmatrix}
y_1\\
y_2\\
y_3\\
y_4\\
x_a\\
x_b\\
x_s^+\\
x_s^-\\
x_t
\end{pmatrix}$ is $
\begin{pmatrix}
	0\\
	0\\
	2\\
	5\\
	0\\
	0\\
	0\\
	0\\
	0\\
\end{pmatrix} + \lambda \begin{pmatrix}
0\\
0\\
0\\
\frac{1}{10}\\
3\\
1\\
0\\
0\\
0\\
\end{pmatrix}  $
	
\section{Problem 3: Short questions on the Simplex Method}
	For each of the following statements, decide if it is true or false and argue why (i.e., give a proof or find a counterexample)
	\subsection{If in a legal pivoting step of the Simplex Method, a variable leaves the basis, there is no legal pivoting step that makes the same variable re-enter the basis in the very next step.}
	Assumption: Entering the basis can only be done via pivoting steps, not re-interpreting the tableau.
	
	\textbf{true}, for a variable to re-enter the basis in the very next step, the pivot-element has to have the same $i,k$ as before. In a legal pivoting step the corresponding $c_k$ will change to a positive value ($c_k' = -\frac{c_k}{A_{ik}}$ with $c_k < 0$ by definition of a legal pivot and $A_{ik} > 0$ by definition of a legal pivot $\Rightarrow c_k' > 0$) and can therefore not be used for the next pivoting step.
	\subsection{If in every potential pivot column of a feasible simplex tableau, there is a legal pivot, then the objective function is bounded.}
	\textbf{false}, sometimes it requires more than 1 step to get to the unbounded vertex. \textbf{PROOF FOLLOWS}
	\subsection{If there exists a legal pivot in a feasible simplex tableau, the current basic solution is not optimal.}
	\textbf{false}, in a degenerated tableau there are infinite legal pivots (cycling problem)
	\subsection{Consider a feasible simplex tableau whose corresponding basic solution is optimal. Then, the entries of the tableau that correspond to coefficients of the objective function are all non-negative.}
	\textbf{maybe, i think no} in degenerated
	\subsection{The number of basic feasible solutions of a system of linear equalities is finite.}
	\textbf{true}
	
\end{document}