\documentclass[12pt, a4]{article}
\usepackage[margin=2cm]{geometry}
\usepackage{parskip}
\usepackage{nameref}
\usepackage{enumitem}
\usepackage{booktabs}
\usepackage{tabularx}
\usepackage{hyperref}
\usepackage[tiny]{titlesec}

\usepackage{amsmath}
\usepackage{amssymb}
\usepackage{bm}
\usepackage[weather]{ifsym}

\usepackage{fancyhdr}
\usepackage{titling}

\usepackage{pgfplots}
\pgfplotsset{compat=1.16}
\usetikzlibrary{decorations.pathreplacing,positioning, shapes,arrows,chains,graphs,calc}


\usepackage{xcolor}
\usepackage{graphicx}
\usepackage{fancyvrb}
\usepackage{listings}
\usepackage{bm}
\usepackage{xcolor}
\usepackage{optidef}

\usepackage{pifont} % for cmark and xmark
\newcommand{\cmark}{\ding{51}}%
\newcommand{\xmark}{\ding{55}}%
\newcommand{\checkedbox}{\rlap{$\square$}{\raisebox{2pt}{\hspace{1pt}\cmark}}}
%\newcommand{\crossedBox}{\rlap{$\square$}{\large\hspace{1pt}\xmark}}


\xdefinecolor{gray}{rgb}{0.4,0.4,0.4}
\xdefinecolor{blue}{RGB}{58,95,205}
\xdefinecolor{darkgreen}{RGB}{0,100,0}

\newcommand{\lightgray}{black!30}

\newcommand{\plotDomain}{-1:8}

\newcommand{\addPlotLDownCoords}[1]{
	\addplot[mark=none, domain=\plotDomain, color=\lightgray,
	decoration={border,segment length=1mm,amplitude=1.5mm,angle=-135},
	postaction={decorate}
	] coordinates {#1};
	\addplot[mark=none, domain=\plotDomain] coordinates {#1};
}

\newcommand{\addPlotLDown}[1]{
	\addplot[mark=none, domain=\plotDomain, color=\lightgray,
	decoration={border,segment length=1mm,amplitude=1.5mm,angle=-135},
	postaction={decorate}
	] {#1};
	\addplot[mark=none, domain=\plotDomain] {#1};
}

\newcommand{\addPlotRUpCoords}[1]{
	\addplot[mark=none, domain=\plotDomain, color=\lightgray,
	decoration={border,segment length=1mm,amplitude=1.5mm,angle=135},
	postaction={decorate}
	] coordinates {#1};
	\addplot[mark=none, domain=\plotDomain] coordinates {#1};
}

\newcommand{\addPlotRUp}[1]{
	\addplot[mark=none, domain=\plotDomain, color=\lightgray,
	decoration={border,segment length=1mm,amplitude=1.5mm,angle=135},
	postaction={decorate}
	] {#1};
	\addplot[mark=none, domain=\plotDomain] {#1};
}

\lstset{% setup listings
	language=R,% set programming language
	basicstyle=\ttfamily\small,% basic font style
	keywordstyle=\color{blue},% keyword style
	commentstyle=\color{gray},% comment style
	numbers=left,% display line numbers on the left side
	numberstyle=\scriptsize,% use small line numbers
	numbersep=10pt,% space between line numbers and code
	tabsize=3,% sizes of tabs
	showstringspaces=false,% do not replace spaces in strings by a certain character
	captionpos=b,% positioning of the caption below
	breaklines=true,% automatic line breaking
	escapeinside={(*}{*)},% escaping to LaTeX
	fancyvrb=true,% verbatim code is typset by listings
	extendedchars=false,% prohibit extended chars (chars of codes 128--255)
	literate={"}{{\texttt{"}}}1{<-}{{$\bm\leftarrow$}}1{<<-}{{$\bm\twoheadleftarrow$}}1
	{~}{{$\bm\sim$}}1{<=}{{$\bm\le$}}1{>=}{{$\bm\ge$}}1{!=}{{$\bm\neq$}}1{^}{{$^{\bm\wedge}$}}1,% item to replace, text, length of chars
	alsoletter={.<-},% becomes a letter
	alsoother={$},% becomes other
	otherkeywords={!=, ~, $, \&, \%/\%, \%*\%, \%\%, <-, <<-, /},% other keywords
	deletekeywords={c}% remove keywords
}



\author{Pascal Lüscher}
\title{Mathematical Optimization – Problem set 11}

\makeatletter
\let\mytitle\@title
\makeatother

\pagestyle{fancy}
\fancyhf{}
\rhead{
	\mytitle\\
	\theauthor
}

\rfoot{
	Page: \thepage
}

\renewcommand{\arraystretch}{1.2} % more space in tables
\renewcommand\thesubsection{\thesection.\alph{subsection}}
\titleformat{\section}[hang]{\normalfont\bfseries\itshape}{\textup{\thesubsection}}{1em}{}

\titleformat{\subsection}[hang]{\normalsize\itshape}{\textup{\thesubsection}}{1em}{}[]

\newcommand{\norm}[1]{\lvert #1 \rvert}

\newcolumntype{L}{>{$}l<{$}} % math-mode version of "l" column type
\newcolumntype{R}{>{$}r<{$}} % math-mode version of "l" column type
\newcolumntype{C}{>{$}c<{$}} % math-mode version of "l" column type

\begin{document}
\section{Problem 3: Perfect matchings in three-regular graphs}
Prove that every 3-regular bridgeless graph admits a perfect matching. To this end, recall that a \emph{bridge} in a graph $G=(V,E)$ is an edge $e\in E$ whose removeal from the graph increases the number of connected components by one. A graph is \emph{bridgeless} if it contains no bridge. Moreover, a graph is $3$-\emph{regular} if the degree of every vertex is three.
\paragraph{Solution}
We show that the perfect matching polytope is non-empty. The perfect matching polytope is defined by $P(G) = \left \{ x\in R^E_{\geq 0} \left|\begin{matrix}
	x(\delta(v)) = 1 & \forall v \in V \\ 
	x(\delta(S)) \geq 1 & \forall S \subseteq V,& |S|\text{ odd}
\end{matrix}\right. \right\}$. 

Wlog we assume the graph is connected.

Let $x$ be the all $\frac{1}{3}$-vector. We need to show that $x$ is in the polytope $P$.
To be in the polytope, all constraints must be satisfied.

\subparagraph{$\bm{x(\delta(v)) = 1 \quad \forall v \in V}$:}
This constraint is satisfied by the construction of $x$. In a $3$-regular graph, every vertex has $3$ edges and therefore the sum of all outgoing edges is equal to $=1$.

\subparagraph{$\bm{x(\delta(S)) \geq 1 \quad \forall S \subseteq V, |S|\text{ odd}}$:}
To fulfill this constraint there need to be 3 outgoing edges for every subset $S$.
By sake of contradiction we assume there are 0, 1, or 2 outgoing edges. By disproving all of them, there are at least 3 outgoing edges.

Case  $0$:
In the lecture we showed that there is no perfect matching in a Graph with odd number of vertices. Since we assumed that $x$ is in $P$ this case will never happen. \Lightning

Case $1$:
If there exists only one edge outgoing of S, that edge becomes a bridge. By definition $G$ is bridgeless. \Lightning

Case $2$:
If we sum up all the vertices in $G$ we get $3\cdot|V|$. Expressed with respect to $S$ we can also say we have $2\cdot e +2 \quad  \forall e\in S$. But $3\cdot|V|$ is odd and $2\cdot e +2$ is even. \Lightning

Therefore the number of outgoing edges have to be at least $3$ which fulfills the constraint.


\end{document}